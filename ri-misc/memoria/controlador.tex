\datos{color}{}{}

\section{Controlador PID}

Una vez generada una trayectoria articular dinámicamente factible, es necesario disponer de un
controlador capaz de seguirla con precisión. En este trabajo se ha implementado un controlador
PID por articulación \cite{barrientos2007fundamentos}, combinado con el modelo dinámico del robot
para generar el par de control adecuado en cada instante. \\

El objetivo del controlador es minimizar el error entre la trayectoria deseada
$\{q_d(t),\dot{q}_d(t),\ddot{q}_d(t)\}$ y la trayectoria real del robot
$\{q(t),\dot{q}(t)\}$. Para ello, se define el error articular:



\[
e(t) = q_d(t) - q(t), \qquad \dot{e}(t) = \dot{q}_d(t) - \dot{q}(t),
\]



y la acción PID se aplica sobre la aceleración de referencia:



\[
\ddot{q}_{\text{ref}}(t) =
\ddot{q}_d(t) + K_p\,e(t) + K_d\,\dot{e}(t) + K_i\!\int e(t)\,dt.
\]



A partir de esta aceleración de referencia, el par de control se obtiene mediante la ecuación
dinámica inversa del robot:



\[
\tau_{\text{PID}}(t) =
M(q)\,\ddot{q}_{\text{ref}}(t) +
C(q,\dot{q})\,\dot{q}(t) +
g(q),
\]



donde $M(q)$ es la matriz de inercia, $C(q,\dot{q})$ el término centrífugo–coriólis y $g(q)$
el vector de gravedad. Este enfoque permite compensar la dinámica del manipulador y mejorar
significativamente el seguimiento de la trayectoria. \\

Para simular condiciones no ideales, se añade una perturbación externa arbitraria $\tau_{\text{pert}}(t)$,
de modo que el par total aplicado es:
\[
\tau_{\text{total}}(t) = \tau_{\text{PID}}(t) + \tau_{\text{pert}}(t).
\]


La evolución real del robot se obtiene integrando la dinámica directa:
\[
\ddot{q}(t) =
M^{-1}(q(t))
\left(\tau_{\text{total}}(t) -
C(q(t),\dot{q}(t))\,\dot{q}(t) -
g(q(t))\right).
\]

Aunque el diseño detallado de controladores PID y su sintonización queda fuera del alcance de esta asignatura,
en este trabajo se ha implementado con fines didácticos. De hecho, los fabricantes, por seguridad, acostumbran a proporcionar el controlador del robot como una ``caja negra'' 
inalterable. Por tanto, esta sección es meramente académica, ya que el robot real trae su propio controlador 
sintonizado. \\

En las figuras \ref{fig:traj1}-\ref{fig:traj6} 
se presentan las trayectorias calculadas y reales seguidas por el robot, donde se observa que las articulaciones
4, 5 y 6 son más difíciles de controlar y requieren una correcta sintonización de constantes $K_p$, $K_d$ y $K_i$. 
En las figuras \ref{fig:traj_art} y \ref{fig:traj_cart} se muestran las trayectorias articulares y cartesiana 
calculadas, en la \ref{fig:vel_art} y \ref{fig:acc_art} las velocidades y articulaciones articulares calculadas y 
en la \ref{fig:torque_all} el par articular calculado durante la trayectoria.\\

La implementación se encuentra en el código \ref{lst:controller}.


% --- Fila 1 ---
\begin{figure}[H]
    \centering
    \begin{minipage}[c]{0.48\textwidth}
        \centering
        \includegraphics[width=\textwidth]{Images/controlador/traj1.png}
        \caption{Par articular en la articulación 1.}
        \label{fig:traj1}
    \end{minipage}
    \hfill
    \begin{minipage}[c]{0.48\textwidth}
        \centering
        \includegraphics[width=\textwidth]{Images/controlador/traj2.png}
        \caption{Par articular en la articulación 2.}
        \label{fig:traj2}
    \end{minipage}
\end{figure}

% --- Fila 2 ---
\begin{figure}[H]
    \centering
    \begin{minipage}[c]{0.48\textwidth}
        \centering
        \includegraphics[width=\textwidth]{Images/controlador/traj3.png}
        \caption{Par articular en la articulación 3.}
        \label{fig:traj3}
    \end{minipage}
    \hfill
    \begin{minipage}[c]{0.48\textwidth}
        \centering
        \includegraphics[width=\textwidth]{Images/controlador/traj4.png}
        \caption{Par articular en la articulación 4.}
        \label{fig:traj4}
    \end{minipage}
\end{figure}

% --- Fila 3 ---
\begin{figure}[H]
    \centering
    \begin{minipage}[c]{0.48\textwidth}
        \centering
        \includegraphics[width=\textwidth]{Images/controlador/traj5.png}
        \caption{Par articular en la articulación 5.}
        \label{fig:traj5}
    \end{minipage}
    \hfill
    \begin{minipage}[c]{0.48\textwidth}
        \centering
        \includegraphics[width=\textwidth]{Images/controlador/traj6.png}
        \caption{Par articular en la articulación 6.}
        \label{fig:traj6}
    \end{minipage}
\end{figure}


\begin{figure}[H]
    \centering

    % --- Izquierda ---
    \begin{minipage}[c]{0.48\textwidth}
        \centering
        \includegraphics[width=\textwidth]{Images/controlador/traj_art.png}
        \caption{Trayectoria articular.}
        \label{fig:traj_art}
    \end{minipage}
    \hfill
    % --- Derecha ---
    \begin{minipage}[c]{0.48\textwidth}
        \centering
        \includegraphics[width=\textwidth]{Images/controlador/traj_cart.png}
        \caption{Trayectoria cartesiana.}
        \label{fig:traj_cart}
    \end{minipage}

\end{figure}


\begin{figure}[H]
    \centering

    % --- Izquierda ---
    \begin{minipage}[c]{0.48\textwidth}
        \centering
        \includegraphics[width=\textwidth]{Images/controlador/vel_art.png}
        \caption{Velocidades articulares.}
        \label{fig:vel_art}
    \end{minipage}
    \hfill
    % --- Derecha ---
    \begin{minipage}[c]{0.48\textwidth}
        \centering
        \includegraphics[width=\textwidth]{Images/controlador/acc_art.png}
        \caption{Aceleraciones articulares.}
        \label{fig:acc_art}
    \end{minipage}

\end{figure}

\begin{figure}[H]
    \centering
    \includegraphics[width=0.95\textwidth]{Images/controlador/par_art.png}
    \caption{Par articular total para las seis articulaciones durante el seguimiento.}
    \label{fig:torque_all}
\end{figure}